\documentclass[12pt]{article}
\usepackage{amsmath}
\usepackage[shortlabels]{enumitem}
\usepackage{tcolorbox}

\usepackage{graphicx}
\usepackage{amsthm}
\usepackage{amssymb}
\usepackage{thmtools}
\usepackage{amsthm}
\usepackage{algorithm}
\usepackage{algpseudocode}
\usepackage{xspace}
\usepackage{CJKutf8}
\usepackage[hidelinks,pdfencoding=auto,psdextra]{hyperref}

\usepackage{graphicx}
\usepackage{pythonhighlight}


%\renewcommand{\thepage}{}

\textwidth=6.7in
\textheight=8.4in
\oddsidemargin=-.3in
\evensidemargin=-.1in
\topmargin=-.3in

\newcommand{\Bskip}{\vspace*{.3in}}
\newcommand{\Solution}{\ \\ \textbf{Solution:} }
\newcommand{\Answer}{\ \\ \textbf{Answer:} }
\newtheorem{theorem}{Theorem}
\newtheorem{remark}{Remark}
\newtheorem{hint}{Hint}

\DeclareMathOperator*{\E}{\mathbb{E}}
\newcommand{\Var}{\mathsf{Var}}
\newcommand{\Cov}{\mathsf{Cov}}
\newcommand{\Covar}{\mathsf{Covar}}
\declaretheoremstyle[headfont=\bf]{normalhead}
\declaretheorem[style=normalhead]{problem}
\def\ie{\textit{i.e.}\xspace}
\def\eg{\textit{e.g.}\xspace}

\begin{document}

\noindent
\hspace*{.2in} COMP7102P Fall 2022
\hfill Homework 1\\
\begin{CJK}{UTF8}{gbsn}
\hspace*{.2in} \textcolor{red}{BA22011024 \Name{} \ChineseName} \hfill due: Sept 22, 09:30
\end{CJK}

\bigskip
%When asked to design an algorithm with certain properties, you must prove that these properties hold for your algorithm.

\begin{problem}[30 points]
Prove the following claims and show your calculations.
\begin{enumerate}[(a)]
\item Prove that $\E[Y] \le \E[Y^2]^\frac{1}{2}$ for any real random variable $Y$. Moreover, show this implies $\E[|X - \E[X]|] \le \sqrt{\Var(X)}$.

\Answer
\begin{definition}[The variance of a real random variable $Y$]
    \begin{align}\label{eq:var}
        \Var(Y) = \E[(Y - E(Y))^2] = \E[Y^2 - 2Y\E(Y) + \E(Y)^2] = \E(Y^2) - \E(Y)^2
    \end{align}
    where $\Var(Y) \ge 0$.
\end{definition}

According to definition 1, we have  $\E[Y^2] \ge \E[Y]^2$. By taking the square root of both sides, we have 
\begin{align}\label{eq:var1}
    \E[Y] \le \E[Y^2]^\frac{1}{2}
\end{align}

By taking the square root of Eq. \eqref{eq:var} and replace $Y$ with $X$, we can obtain $\sqrt{\Var(X)} = \sqrt{\E[(X - E(X))^2]} = \E[(X - E(X))^2]^\frac{1}{2}$. According to Eq. \eqref{eq:var1}, we have $\E[X - E(X)] \le \E[(X - E(X))^2]^\frac{1}{2} = \sqrt{\Var(X)}$, which is consistent with the conclusion of problem (a).

\item For $n$ independent variables $X_1, \cdots, X_n$, prove that $\Var(X_1 + X_2 + \cdots + X_n) = \Var(X_1) + \Var(X_2) + \cdots + \Var(X_n)$.

\Answer

Given the definition of variance, we have
\begin{align} \label{eq:1-b}
\notag \Var(X_1 + X_2 + ... + X_n) &= \E[[(X_1 + X_2 + \cdots + X_n) - \E(X_1 + X_2 + \cdots + X_n)]^2]\\
\notag & = \E[[(X_1 - \E(X_1)) + (X_2 - \E(X_2)) + ... + (X_n - \E(X_n))]^2] \\
\notag & = \E[\sum_{i=1}^{n}{(X_i - \E(X_i))^2} + 2\sum_{i=1}^{n-1}\sum_{j=i+1}^{n}{(X_i - \E(X_i))(X_j - \E(X_j))}] \\
\notag & = \sum_{i=1}^{n}{\E[(X_i - \E(X_i))^2]} + 2\sum_{i=1}^{n-1}\sum_{j=i+1}^{n}{\E[(X_i - \E(X_i))(X_j - \E(X_j))]} \notag \\
& = \sum_{i=1}^{n}{\Var(X_i)} + 2\sum_{i=1}^{n-1}\sum_{j=i+1}^{n}{\Cov(X_i, X_j)}
\end{align}

Note that, $X_1, ..., X_n$ are independent variables, we have $\Cov(X_i, X_j) = 0$. As a result, we can obtain $\Var(X_1 + X_2 + \cdots + X_n) = \Var(X_1) + \Var(X_2) + \cdots + \Var(X_n)$.


\item Consider $d$ independent random coins $Z_1, ..., Z_d \in \{\pm 1\}$ where each $Z_i$ is 1 or -1 with probability $1/2$ separately. We define $n = 2^d - 1$ random variables as follows: 
For each non-empty subset $S \subseteq [n]$, we define $X_s = \prod_{i \in S} Z_i$. 

For example when $d = 3$, there are 7 random variables $X_1 = Z_1, X_2 = Z_2, X_3 = Z_3, X_{1,2} = Z_1 \cdot Z_2, X_{1,3} = Z_1 \cdot Z_3, X_{2,3} = Z_2 \cdot Z_3$
and $X_{1,2,3} = Z_1 \cdot Z_2 \cdot Z_3$. 

Calculate $\Var(\sum_{S}X_S)$ and compare this with part (b).

\Answer

For each coin $Z_i$, we have $\E[Z_i] = 1 \times \frac{1}{2} + (-1) \times \frac{1}{2} = 0$ and $\E[Z_i^2] = 1^2 \times \frac{1}{2} + (-1)^2 \times \frac{1}{2} = 1$.

Since each coin is independent, we have 
\begin{equation} \label{eq:1-3-1}
    \E[X_S] = \E[\prod_{i \in S}Z_i] = \prod_{i \in S}\E[Z_i] = 0
\end{equation}

\begin{equation} \label{eq:1-3-2}
    \E[X_S^2] = \E[\prod_{i \in S}Z_i^2] = \prod_{i \in S}\E[Z_i^2] = 1
\end{equation}

Combining Eq. \eqref{eq:1-3-1} and Eq. \eqref{eq:1-3-2}, we have

\begin{align} \label{eq:1-3-3}
\Var(\sum_{S}X_S) &= \Var(X_1 + X_2 + \cdots + X_n) \notag \\
&= \E[(X_1 + X_2 + \cdots + X_n)^2] - \E[X_1 + X_2 + \cdots + X_n]^2  \notag  \\
&= \sum_{i=1}^n\E[X_i^2] + 2\sum_{i=1}^{n-1}\sum_{j=i+1}^{n}\E[X_iX_j] - \sum_{i=1}^n\E[X_i]^2 - 2\sum_{i=1}^{n-1}\sum_{j=i+1}^{n}\E[X_i]\E[X_j] \notag \\
&= n + 0 - 0 - 0 \notag  \\
&= n
\end{align}

Moreover, we can calculate $\sum_{S}\Var(X_S)$ as follows:

\begin{align} \label{eq:1-3-4}
\sum_{S}\Var(X_S) &= \Var(X_1) + \Var(X_2) + \cdots + \Var(X_n) \notag \\
&= (\E[X_1^2] - \E[X_1]^2) + (\E[X_2^2] - \E[X_2]^2) + \cdots + (\E[X_n^2] - \E[X_n]^2) \notag  \\
&= (1 - 0) + (1 - 0) + \cdots + (1 - 0) \notag  \\
&= n
\end{align}

According to Eq. \eqref{eq:1-3-3} and Eq. \eqref{eq:1-3-4}, we can conclude that $\Var(\sum_{S}X_S) = \sum_{S}\Var(X_S)$, which is consistent with the conclusion in part (b).

\end{enumerate}
\end{problem}

~\\

\begin{problem}[20 points.] Consider a random distribution $D$ over those bins $[n] = \{1, 2, ..., n\}$. We define its collision probability to be
\begin{equation}
\begin{aligned}
\mathbf{Pr}_{a \sim D, b \sim D}{[a = b]} \notag
\end{aligned}
\end{equation}
where $a$ and $b$ are drawn from D independently.  \\
Prove that the uniform distribution has the smallest collision probability among all distributions.
This is the reason why we only consider uniform distribution over n bins in hash functions.

\begin{hint}
Define a random variable $X$ (depends on $D$) such that the collision probability is equal.
\end{hint}

\Answer

We first define a random variable $X$ with the uniform distribution, \ie, $P(X = i) = \frac{1}{n}$, $\forall i \in \{1, 2, ..., n\}$. Then, we define a set of random variables $Y=\{y_1, y_2, ..., y_n\}$ with the distribution $D$. We let $p_i$ denote the probability of selecting $y_i$ from the set $Y$, and we have $\sum_{i=1}^n p_i =1$.

The collision probability of $Y$ can be rewrote to $\sum_{i=1}^n p_i^2$, which denotes the probability of independently drawing two identical variables from $Y$.

\begin{definition}[Average Inequality]
    \begin{align}\label{eq:average}
        \sqrt{\frac{\sum_{i=1}^n x_i^2}{n}} \geq \frac{\sum_{i=1}^n x_i}{n}
    \end{align}
\end{definition}

According to Eq. \eqref{eq:average}, we have 

\begin{align}\label{eq:2-1}
    \sum_{i=1}^n p_i^2 \geq n \cdot (\frac{\sum_{i=1}^2 p_i}{n})^2= n \cdot (\frac{1}{n})^2 = \frac{1}{n}
\end{align}

Eq. \eqref{eq:2-1} denotes that the collision probability can achieve the minimum value, if and only if $p_i=\frac{1}{n}$. As a result, we can conclude that the uniform distribution has the smallest collision probability among all distributions.
\end{problem}

~\\

\begin{problem}[Birthday paradox 10 points.]
Suppose everybody's birthday is a uniform random number in $\{1, 2, \cdots, 365\}$ independently.
Now we wanna ask $m$ persons' birthday such that with probability more than $\frac{1}{2}$, two of them will have the same birthday. \\
\hspace*{.2in} Show the best estimation of $m$.

\begin{hint}
Think of this as balls into bins with $n = 365$ bins.
\end{hint}

\Answer

The problem can be converted to \textit{at least how many persons do not have the same birthday with a probability of no more than $\frac{1}{2}$ ?}.

\begin{align}\label{eq:3-1}
P= (1 - \frac{1}{n})(1 - \frac{2}{n}) \cdots (1 - \frac{m - 1}{n}) \le \frac{1}{2}
\end{align}

We now give the inequality $e^x \ge 1 + x, \forall x \ge 0$, and apply it to Eq. \eqref{eq:3-1}.

\begin{align}
    \notag P&= (1 - \frac{1}{n})(1 - \frac{2}{n}) \cdots (1 - \frac{m - 1}{n}) \\ 
    \notag & \le e^{-\frac{1}{n}}e^{-\frac{2}{n}} \cdots e^{-\frac{m-1}{n}}
    = e^{-\frac{1}{n}-\frac{2}{n} \cdots -\frac{m-1}{n}} \\ 
& = e^{-\frac{m \cdot (m-1)}{2n}} \le \frac{1}{2}
\end{align}

Further, we have:

\begin{align}
    \notag e^{-\frac{m(m-1)}{2n}} \le \frac{1}{2}  
    & \Rightarrow -\frac{m(m-1)}{2n} \le -ln2, \\  
    \notag &\Rightarrow m^2 - m - 2n ln 2 \ge 0 \\ 
&\Rightarrow m \ge \frac{1 + \sqrt{1 + 8n ln2}}{2}
\end{align}

Given $n = 365$, we need to ask at least 23 people, \ie, $m=23$, such that with probability more than $\frac{1}{2}$, two of them will have the same birthday.

\end{problem}

~\\
\begin{problem}[20 points.]
Let us finish the calculation about variance left in lecture 1. Consider the simple hash scheme: for a perfectly random hash $h$, we throw ball $1$, . . . , $n$ into $h(1)$, . . . , $h(n)$ directly. Fix a load $k$ and consider the indicator variable $Z_i$ that denotes whether bin $i$ has $k$ balls or not.

\begin{enumerate}[(a)]
\item Prove that $|\Covar(Z_i,Z_j)|=O(1/n)$ for any constant $k$.

\Answer

According to lecture 1, when $k<\frac{3\log n}{\log\log n}$, the expected number of bins with k is $\frac{n}{ek!}$. We have

\begin{align} \label{eq:4-1-1}
    \begin{cases}
        \E[Z_i] = P(Z_i=1)=\frac{\tbinom{\frac{n}{ek!}}{1}}{\tbinom{n}{1}} , \forall i \in \{1, 2, ... , n\}\\
        \E[Z_i^2] = P(Z_i=1)=\frac{\tbinom{\frac{n}{ek!}}{1}}{\tbinom{n}{1}} , \forall i \in \{1, 2, ... , n\} \\
        \E[Z_i Z_j] = P(Z_i=1, Z_j=1)=\frac{\tbinom{\frac{n}{ek!}}{2}}{\tbinom{n}{2}} , \forall i \neq j
    \end{cases}
\end{align}

Given the definition of covar, we have
\begin{align} \label{eq:4-1-2}
    \notag |\Cov(Z_i, Z_j)|&=|\E[Z_i Z_j] - \E[Z_i]\E[Z_j]|\\
    \notag &=|\frac{\tbinom{\frac{n}{ek!}}{2}}{\tbinom{n}{2}} - [\frac{\tbinom{\frac{n}{ek!}}{1}}{\tbinom{n}{1}}]^2|\\
    \notag & = |\frac{\frac{n}{ek!}{(\frac{n}{ek!}-1)}}{n(n-1)} - (\frac{1}{ek!})^2|\\
    & =| \frac{\frac{1}{ek!}-1}{ek!}\frac{1}{n-1}| = O(\frac{1}{n})
\end{align}

When $k \ge \frac{3\log n}{\log\log n}$, the expected number of bins with k is 0.

Therefore, we can prove that $ |\Cov(Z_i,Z_j)| =  O(\frac{1}{n})$.

\item Show that $ \Var (Z_1 + \dots+Z_n)=O(n)$.

\Answer

From Eq. \eqref{eq:1-b}, we have $\Var(Z_1 + \cdots + Z_n) = \sum_{i=1}^{n}{\Var(Z_i)} + 2\sum_{i=1}^{n-1}\sum_{j=i+1}^{n}{\Cov(Z_i, Z_j)}$.

Considering the conclusion in part(a), we have

\begin{align}
    \notag & \frac{n}{ek!} + n(n-1)[(\frac{1}{ek!})^2 + O(\frac{1}{n})] -(\frac{n}{ek!})^2 \\
    & = \frac{n}{ek!} - \frac{n}{(ek!)^2} +n(n-1)O(\frac{1}{n})= O(n)
\end{align}

\item Apply the Chebyshev's inequality to conclude that with prob $1-n^{-1/2}$, the number of bins with exact $k$ balls is $\frac{n}{k! \cdot e } ± O(n^{0.75})$.

\Answer

From lecture 1, we have $P[|X-\E[X]| > C\sigma] \leq \frac{1}{C^2}$, where $\sigma = \sqrt{\Var(X)}$ and $C > 1$. Let $\frac{1}{C^2} = n^{-1/2}$, thereby $C = \sqrt{\frac{1}{n^{-1/2}}}$.
\begin{align}
    \notag C\sigma &= \sqrt{\frac{1}{n^{-1/2}}}\sqrt{O(n)} \\
    & = \sqrt{O(n)n^{1/2}}= O(n^{0.75})
\end{align}

\end{enumerate}
\end{problem}

~\\

\begin{problem}[20 points]
Consider the following generalization of Power of 2 choices to $d > 2$ choices: \\
1. Prepare $d$ perfectly random hash functions $h_1, \cdots, h_d : U -> [n]$. \\
2. For each ball $A$, allocate it into the bin among his $d$ choices $\{h_1(A), \cdots, h_d(A)\}$ with the lightest load at this moment. \\

Suppose we are throwing $n$ balls into $n$ bins. Modify the proof sketch in lecture 2 to show that: With probability $1 - n^{-2}$, 
the max-load is $\log_d{\log n} + O(1)$ instead of $\log_2{\log n} + O(1)$

\Answer

Base case: $m_3 \le \frac{n}{3}$;

Bounding $m_4$: since $\E[m_4] \le \frac{n}{3^d}$, it implies $m_4 \le 1.1 \cdot \frac{n}{3^d}$ w.h.p;

Bounding $m_5$: $m_5 \le 1.1n \cdot (\frac{m_4}{n})^d = 1.1^{d+1} \cdot \frac{n}{3^{d^2}}$;

$\cdots$

Bounding $m_l$: $m_l \le 1.1^{d^{l-3}-1} \cdot \frac{n}{3^{d^{l-3}}} \le \frac{n}{2^{d^{l-3}}}$.

Taking the logarithm of both sides of the inequality, we have:

\begin{align}
    \notag \frac{n}{2^{d^{l-3}}} \le 1  & \Rightarrow n \le 2^{d^{l-3}} \\
    \notag & \Rightarrow  \log_2 n \le d^{l-3} \\
    & \Rightarrow \log_d{\log_2 n} \le l-3
\end{align}

As a result, we can prove the max-load is $\log_d{\log n} + O(1)$.

\end{problem}

\end{document}
