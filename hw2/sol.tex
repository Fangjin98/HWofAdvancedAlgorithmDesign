\documentclass[12pt]{article}
\usepackage{amsmath}
\usepackage[shortlabels]{enumitem}
\usepackage{tcolorbox}

\usepackage{graphicx}
\usepackage{amsthm}
\usepackage{amssymb}
\usepackage{thmtools}
\usepackage{amsthm}
\usepackage{algorithm}
\usepackage{algpseudocode}
\usepackage{xspace}
\usepackage{CJKutf8}
\usepackage[hidelinks,pdfencoding=auto,psdextra]{hyperref}

\usepackage{graphicx}
\usepackage{pythonhighlight}


%\renewcommand{\thepage}{}

\textwidth=6.7in
\textheight=8.4in
\oddsidemargin=-.3in
\evensidemargin=-.1in
\topmargin=-.3in

\newcommand{\Bskip}{\vspace*{.3in}}
\newcommand{\Solution}{\ \\ \textbf{Solution:} }
\newcommand{\Answer}{\ \\ \textbf{Answer:} }
\newtheorem{theorem}{Theorem}
\newtheorem{remark}{Remark}
\newtheorem{hint}{Hint}

\DeclareMathOperator*{\E}{\mathbb{E}}
\newcommand{\Var}{\mathsf{Var}}
\newcommand{\Cov}{\mathsf{Cov}}
\newcommand{\Covar}{\mathsf{Covar}}
\declaretheoremstyle[headfont=\bf]{normalhead}
\declaretheorem[style=normalhead]{problem}
\def\ie{\textit{i.e.}\xspace}
\def\eg{\textit{e.g.}\xspace}

\begin{document}

\noindent
\hspace*{.2in} COMP7102P Fall 2022
\hfill Homework 2\\
\begin{CJK}{UTF8}{gbsn}
    \hspace*{.2in} \textcolor{red}{BA22011024 \Name{} \ChineseName} \hfill due: Oct. 13, 09:30
\end{CJK}

\bigskip
%When asked to design an algorithm with certain properties, you must prove that these properties hold for your algorithm.


\begin{problem}[15 points]
Recall the sequence of random variables defined in part 4 of Problem 1 in the 1st homework: Let $Z = \sum_{S}X_S$ in this problem. Given the fact $\Var(Z) = n$, answer the following two questions.

\begin{enumerate}[(a)]
\item Show that for any $C > 1$, $\mathbf{Pr}[|Z| \ge C\sqrt{n}] \le \frac{1}{C^2}$ by Chebyshev's inequality.

\Answer
According to the definition of $Z_i$, we can obtain $\E[Z_i] = 1 \cdot \frac{1}{2} + (-1) \cdot \frac{1}{2} = 0, \E[Z_i^2] = 1^2 \times \frac{1}{2} + (-1)^2 \times \frac{1}{2} = 1, \forall i \in \{1, 2, \cdots, d\}$,
Since each coin $Z_i$ is independent, we have:
\begin{align}
    \E[X_S] = \E[\prod_{i \in S} Z_i] = \prod_{i \in S} \E[Z_i] = 0\\
    \E[Z] = \E[\sum_{S} X_S] = \sum_{S}\E[X_S] = 0
\end{align}
Let $\epsilon = C \cdot \sqrt{n}$. Since $C \ge 1$ and $n=2^d -1$, we have $\epsilon \ge 0$. According to Chebyshev's inequality, we have 
\begin{align} \label{eq:1}
    \mathbf{Pr}[|Z - \E[Z]| \geq \epsilon] \le \frac{\Var{Z}}{\epsilon^2}
\end{align}
Given the fact that $\E[Z] = 0$ and $\Var(Z) = n$, we can convert Eq. \eqref{eq:1} to 
\begin{align}
    &\notag\mathbf{Pr}[|Z| \geq C \cdot \sqrt{n}] \le \frac{n}{C^2 \cdot n} \\
    \Rightarrow & \mathbf{Pr}[|Z| \geq C \cdot \sqrt{n}] \le \frac{1}{C^2}
\end{align}


\item Describe the distribution of $Z$ (by computing $\mathbf{Pr}[Z = l]$ for every $l = 0, 1, \cdots, n$) and compare it with the concentration bound shown in (a).

\Answer
We use $a$ and $b$ to denote the number of 1 and -1 in $\{Z_i\}$, separately, where $ a+b = d$.
We first consider the number of 1 in $\{Z_i\}$ when $d=3$.
\begin{itemize}
    \item \textbf{a=0}: $Z= -1-1-1+1+1+1-1 = -1$. Only $C_3^0=1$ kind.
    \item \textbf{a=1}: Supposing $Z_1=1$, $Z= 1-1-1-1-1+1+1 =-1$. $C_3^1=3$ kinds.
    \item \textbf{a=2}: Supposing $Z_1=1, Z_2=1$, $Z= 1+1-1+1-1-1-1 =-1$. $C_3^2=3$ kinds.
    \item \textbf{a=3}: $Z= 1+1+1+1+1+1+1 =7$. $C_3^3=1$ kind.
\end{itemize} 

From the above analysis, we can conlude that the value of $Z$ can be calculated as following:
\begin{equation} \label{eq:2}
    Z = C_a^1 - C_b^1 + C_a^2 - C_a^1 \cdot C_b^1 + C_b^2 + \dots
\end{equation}

According to the binomial theorem, we can convert Eq. \eqref{eq:2} to 
\begin{equation}
    Z= (1+x)^a \cdot (1-x)^b-1
\end{equation}

When $a=d$, we have $Z= C_a^1+C_a^2+\dots+C_a^a=2^a-1 = 2^d-1=n$. Let $x=1$, we have $Z=-1$.

Thus, we can conclude that $\mathbf{Pr}[Z = 2^d - 1] = \frac{1}{2^d} = \frac{1}{n+1}$ and $\mathbf{Pr}[Z = -1] = 1 - \frac{1}{2^d} = 1 - \frac{1}{n+1}$.

When $Z = -1$, $\mathbf{Pr}[|Z| \ge C\sqrt{n}] = \mathbf{Pr}[1 \ge C\sqrt{n}] = 0 \le \frac{1}{C^2}$ (for any $C > 1$) is satisfied. \\
When $Z = 2^d - 1 = n$, if $1 < C \le \sqrt{n}$, $\mathbf{Pr}[|Z| \ge C\sqrt{n}] = \frac{1}{n+1} \le \frac{1}{C^2}$ is satisfied; if $C > \sqrt{n}$, $\mathbf{Pr}[|Z| \ge C\sqrt{n}] = \mathbf{Pr}[n \ge C\sqrt{n}] = 0 \le \frac{1}{C^2}$ is satisfied. 

\end{enumerate}
\end{problem}

\begin{problem}[30 points]
Recall that a hash family $H$ is 4-wise independent if and only if it satisfies for any four distinct balls a; b; c; and d, \\
$(h(a), h(b), h(c), h(d))$ gives the uniform distribution over $[n] \times [n] \times [n] \times [n]$ for a random $h \sim H$. \\
Similar to the 2nd problem, let us consider $n$ fixed balls into $n$ bins: we sample $h \sim H$ from a 4-wise independent hash family and throw ball $A$ into bin $H(A)$ directly.

We use $X$ to denote the load of the 1st bin under a random $h \sim H$. The goal of this problem is to show that 4-wise independence gives better bound on the max-load than pariwise independence.

\begin{enumerate}[(a)]
\item Show $\E[|X - \E[X]|^4] \le 4$.

\Answer
Say the $n$ balls are $\{1,2, \dots, n\}$ such that $X=\sum_{i=1}^n \mathbf{I}(h(i)=1)$, where $\mathbf{I}$ denotes the indicator function. Let $Z_i = \mathbf{I}(h(i)=1)-\E[\mathbf{I}(h(i)=1)]$ in this calculation. As a result, $\E[|X - \E[X]|^4]=\E[(\sum_i Z_i)^4]$. Moreover, we have $\mathbf{Pr}[Z_i = 1 - \frac{1}{n}] = \frac{1}{n}$ and $\mathbf{Pr}[Z_i = -\frac{1}{n}] = 1 - \frac{1}{n}$. Since these $n$ indicator functions are 4-wise independent, $Z_1, Z_2, \dots, Z_n$ are 4-wise independent either.

First, we compute the first 4 moments of $Z_i$ as follows:

\begin{align}
& \E[Z_i] = (1 - \frac{1}{n}) \cdot \frac{1}{n} + (-\frac{1}{n}) \cdot (1 - \frac{1}{n}) = 0 \notag \\
& \E[Z_i^2] = (1 - \frac{1}{n})^2 \cdot \frac{1}{n} + (-\frac{1}{n})^2 \cdot (1 - \frac{1}{n}) = \frac{n-1}{n^2} \notag \\
& \E[Z_i^3] = (1 - \frac{1}{n})^3 \cdot \frac{1}{n} + (-\frac{1}{n})^3 \cdot (1 - \frac{1}{n}) = \frac{(n-1)(n-2)}{n^3} \notag \\
& \E[Z_i^4] = (1 - \frac{1}{n})^4 \cdot \frac{1}{n} + (-\frac{1}{n})^4 \cdot (1 - \frac{1}{n}) = \frac{(n-1)(n^2-3n+3)}{n^4} \notag
\end{align}

Then we expand $\E[(\sum_{i}{Z_i})^4]$ with linearity of expectation as follows:

\begin{align} \label{eq:3}
\E[(\sum_{i}{Z_i})^4] = & \sum_{i}{\E[Z_i^4]} + 6 \cdot \sum_{i<j}{\E[Z_i^2 Z_j^2]} + 24 \cdot \sum_{i<j<k<l}{\E[Z_i Z_j Z_k Z_l]} \notag \\
& + 12 \cdot \sum_{i<j \text{ and } k \notin {i,j}}{\E[Z_i Z_j Z_k^2]} + 4 \cdot \sum_{i \neq j}{\E[Z_i^3 Z_j]}
\end{align}

Since the hash family $H$ is 4-wise independent, $Z_i$ are also 4-wise independent, we have:

\begin{align}
& \E[Z_i^2 Z_j^2] = \E[Z_i^2]\E[Z_j^2] = \frac{(n-1)^2}{n^4} \notag \\
& \E[Z_i Z_j Z_k Z_l] = \E[Z_i]\E[Z_j]\E[Z_k]\E[Z_l] = 0 \notag \\
& \E[Z_i Z_j Z_k^2] = \E[Z_i]\E[Z_j]\E[Z_k^2] = 0 \notag \\
& \E[Z_i^3 Z_j] = \E[Z_i^3]\E[Z_j] = 0 \notag
\end{align}

We plug these into Eq. \eqref{eq:3}:

\begin{align}
\E[(\sum_{i}{Z_i})^4] = & \sum_{i}{\E[Z_i^4]} + 6 \cdot \sum_{i<j}{\E[Z_i^2 Z_j^2]} \notag \\
& = n \cdot \frac{(n-1)(n^2-3n+3)}{n^4} + 6 \cdot \frac{n(n-1)}{2} \cdot \frac{(n-1)^2}{n^4}  \notag \\
& = \frac{4n^3 - 13n^2 + 15n - 6}{n^3} = 4 - \frac{13n^2 - 15n +6}{n^3}
\end{align}

Since $13n^2 - 15n +6 > 0$ holds for any $n \in N^*$, we have $4 - \frac{13n^2 - 15n +6}{n^3} < 4$, \ie $\E[|X - \E[x]|^4] \le 4$.

\item Prove this generalized Chebyshev's inequality (check lecture note 2 for the proof of the Chebyshev's inequality)
\begin{gather*}
\mathbf{Pr}[|X - \E[X]| \ge k] \le \frac{\E[|X - \E[X]|^4]}{k^4} \notag \\
\end{gather*}
and conclude that $\mathbf{Pr}[X \ge i] \le \frac{4}{(i - 1)^4}$ for any $i \ge 3$.

\Answer 
Let $Y = |X - \E[X]|^4$, we know $\mathbf{Pr}[Y \ge \frac{\E[Y]}{\epsilon^4}] \le \epsilon^4$ from Markov's inequality.

Taking the $\frac{1}{4}$ root of $Y$ and $\epsilon$, we can obtain 
\begin{equation}
    \mathbf{Pr}[|X - \E[X]| \ge \frac{1}{\epsilon}\sqrt[4]{\E[Y]}] \le \epsilon^4
\end{equation}
Let $k=\frac{1}{\epsilon}\sqrt[4]{\E[Y]}$, and we can obtain 
\begin{equation}
    \mathbf{Pr}[|X - \E[X]| \ge k] \le \frac{\E[Y]}{k^4} = \frac{\E[|X - \E[X]|^4]}{k^4}
\end{equation}

Since $\E[X] = \sum_{i=1}^{n}{\E[h(i) = b_1]} = 1$. We have 
\begin{equation}
    \mathbf{Pr}[|X - \E[X]| \ge k] = \mathbf{Pr}[X \ge k + 1]\le \frac{\E[|X - \E[X]|^4]}{k^4}
\end{equation}
From part (a) we have $\E[|X - \E[X]|^4] \le 4$, so $\mathbf{Pr}[X \ge k + 1]\le \frac{4}{k^4}$. Let $i=k+1$, we can obtain that $\mathbf{Pr}[X \ge i]\le \frac{4}{(i-1)^4}$ for any $i \ge 3$.


\item Derive a bound on the max-load for 4-wise independence.

\Answer

We use $M$ to denote the max-load, and formulate the probability as follows
\begin{equation}
    \mathbf{Pr}[M \ge k] \le \sum_{i} \mathbf{Pr}[X \ge k] = n \cdot \mathbf{Pr}[X \ge k] \notag
\end{equation}
From (b), we have $ \mathbf{Pr}[X \ge k]\le \frac{4}{(k-1)^4}$. Note that, we want there is no bin has more than $k$ balls with probability $p$, which means

\begin{equation}
     \mathbf{Pr}[M \ge k] = n \cdot  \mathbf{Pr}[X \ge k] \le n \cdot \frac{4}{(k-1)^4} \le 1-p \notag
\end{equation}

As a result, $k \ge 1 + \sqrt[4]{\frac{4n}{1-p}} = \Theta(\sqrt[4]{n})$.

\item What is the max-load of 2-wise independence? Compare it with the bound in (c) and discuss the tradeoff.

\Answer
The maximum load for a pariwise independence is $\Theta(\sqrt{n})$, which is more than the maximum load for a 4-wise independence ($\Theta(\sqrt[4]{n})$).
Therefore, to get a higher concentration than pairwise independence, we employ greater randomness (4-wise independence).
On the one hand, we can use the pariwise independence hash for convenience when $n$ is small. Otherwise, we can employ 4-wise independence to achieve outstanding performance with a lower max-load (or collision probability).


\item If we use $k$-wise independent hash function for a even constant $k > 4$, what is the max-load? No need to show your calculations.

\Answer $\Theta(\sqrt[k]{n})$

\end{enumerate}
\end{problem}

\begin{problem}
    Recall the goal of hash is to solve the dictionary problem. For many applications, all elements in $S$ are fixed (or barely change); and the only operation is \textsc{Lookup}. In this case, there is a much more efficient solution (with a small failure probability).

    Consider $S = \{b_1 , \dots , b_m \}$ with $m$ fixed balls. We prepare $k$ independent perfectly random hash functions $h_1 , \dots , h_k$ and one table $T : [n] \rightarrow \{0, 1\}$. There are two procedures:

\begin{algorithm}[ht]
\begin{algorithmic}[1]
    \Procedure{Initialization}{$S$}
        \State Set all marks in $T$ to be false
        \For{$b_i \in S$}
        \For{$j \in [k]$}
        \State Mark $T[h_j(b_i)]=1$
        \EndFor
        \EndFor
    \EndProcedure

    \Function{Lookup}{$b$}
    \If{$(T[h_1(b_i)]=1) \wedge  (T[h_2(b_i)]=1) \wedge \dots \wedge (T[h_k(b_i)]=1)$}
    \State Return 1
    \EndIf
    \State Return 0
    \EndFunction
\end{algorithmic}
\end{algorithm}

While it is easy to implement and efficient, let us analyze its performance.
\begin{enumerate}[(a)]
    \item Prove that if $b\in S$, \textsc{Lookup}$(b)$ always returns the correct answer 1.
    
    \Answer
    For each $b\in S$, it will return 0 only if there is one $T[h_j(b_i)]=0 ,\forall j \in [k]$. From line 2 and 4-6 of procedure INITIALIZATION, we know that once $T[h_j(b_i)]$ is set to 1, it won't be set to 0. Therefore, if $b\in S$, the procedure will set $T[h_j(b_i)]=1 ,\forall j \in [k]$, \ie, there is no $T[h_j(b_i)]=0 ,\forall j \in [k]$.

    \item But \textsc{Lookup}$(b)$ may return 1 for $b \notin S$, let us analyze this failure probability. Write down $\mathbf{Pr}_{h_1 ,\dots,h_k}[\textsc{Lookup}(b) = 1]$ for $b \notin S$ as a probability of $k, n, m$.
    
    \Answer

    We first consider the situation of hash $h_1$ with one ball $b$, we have 
    \begin{align}
        \begin{cases}
            \mathbf{Pr}[T[h_j(b)]=1]=\frac{1}{n} \\
            \mathbf{Pr}[T[h_j(b)]=0]=1- \frac{1}{n}
        \end{cases}
    \end{align}
     
    Expanding the situation to $k$ hashs, we have
    \begin{equation}
        \mathbf{Pr}_{h_1 ,\dots,h_k}[\textsc{Lookup}(b) = 0]=(1-\frac{1}{n})^k
    \end{equation}

    Now, we consider the situation with $m$ balls, the above equation can be expanded as following
    \begin{equation}
        \begin{cases}
            \mathbf{Pr}_{h_1 ,\dots,h_k}[\textsc{Lookup}(b) = 0]=(1-\frac{1}{n})^{km} \\
            \mathbf{Pr}_{h_1 ,\dots,h_k}[\textsc{Lookup}(b) = 1]=1- (1-\frac{1}{n})^{km}
        \end{cases}
    \end{equation}

    As a result, the failure probability is 
    \begin{align}
        &\mathbf{Pr}_{h_1 ,\dots,h_k}[\textsc{Lookup}(b) = 1] \notag \\
        = &\mathbf{Pr}[T[h_1(b)]=1] \wedge \mathbf{Pr}[T[h_2(b)]=1] \wedge \dots \wedge \mathbf{Pr}[T[h_k(b)]=1] \notag \\
        =&(1- (1-\frac{1}{n})^{km})^k \notag \\
        \approx& (1-e^{-\frac{km}{n}})^k
    \end{align}
     
 
    \item Suppose $n=C \cdot m$ for a constant $C \ge 1$. Find a good choice of $k$ to minimize the probability in (b).
    
    \Answer
    We use $\mathbf{Pr}$ to denote $\mathbf{Pr}_{h_1 ,\dots,h_k}[\textsc{Lookup}(b) = 1]$, by taking logarithm of $P$, we have
    
    \begin{align}
        f(k)&= \ln \mathbf{Pr} = k \ln[1-e^{-\frac{km}{n}}] = k \ln[1-e^{-\frac{k}{C}}] \\
        f'(k) &= ln[1-e^{-\frac{k}{C}}] + k \frac{e^{-\frac{k}{C}}}{C(1-e^{-\frac{k}{C}})}
    \end{align}

    Let $t$ denote $e^{-\frac{k}{C}}$ and we can obtain

    \begin{align}   
        &\ln(1-t) + k \frac{t}{C(1-t)} =0 \notag\\
    \Rightarrow & \frac{\ln(1-t)}{\ln t}= \frac{t}{1-t}  \notag\\
    \Rightarrow &t=\frac{1}{2}
    \end{align}
    
    As a result, $k = C \cdot \ln 2$.


    \item  Simplify the probability in (b) using your choice in (c).
    \Answer
    \begin{equation}
        \begin{aligned}
            \mathbf{Pr}&=\bigg[1-(1-\frac{1}{n})^{km}\bigg]^k \notag\\
            &\approx \bigg[1-e^{-\frac{k}{C}}\bigg]^k\\
            &=\bigg[1-e^{-\frac{Cln2}{C}}\bigg]^{Cln2}\\
            &=(\frac{1}{2})^{Cln2}
        \end{aligned}
    \end{equation}
\end{enumerate}
\end{problem}

\begin{problem}[40 points]
    Consider the following algorithm for heavy hitters: Let $M$ be the number of possible elements and $N$ be the length of the input sequence.
    \begin{enumerate}
        \item \textsc{Initialization}: For $B = 10/\epsilon$, we prepare $\ell = \log_2(N \cdot M/\delta )$ pairwise independent hash $h_1 , \dots , h_\ell : [M] \rightarrow [B]$ and set $CMS[1, \dots , \ell][0, \dots , B-1] = 0$.
        \item \textsc{Process}: For each element $i$, we increase $CMS[j][h_j(i)]$ by 1 for each $j \in [\ell]$.
        \item \textsc{Query}$(i \in [M])$: Output $\widetilde{f}_i = \min\{CMS[1][h_1(i)],CMS[2][h_2(i)],\dots,CMS[d][h_d(i)]\}$ as the approximation of the frequency of element $i$.
    \end{enumerate}
    Please modify the above algorithm or supplement a procedure to implement the following functionality separately (and prove the correctness of your codes).

    \begin{enumerate}[(a)] 
        \item Prove that for every element $i$ and every hash $j \in [\ell]$,
        \begin{equation}
            \notag \mathbf{Pr}[|CMS[j][h_j(i)]-f_j(t)| \geq \epsilon \cdot t] \leq 0.1
        \end{equation}

        \Answer
        For simplicity, we use $Y_i$ to denote $|CMS[j][h_j(i)]-f_j(t)|$ and we have
        \begin{equation}
            Y_i = \sum_{k \in [U]-\{i\}}  \mathbf{I}(h_j(k)=h_j(i)) \cdot f_k(t)
        \end{equation}

        Since $h_j$ is pairwise independent, for each element $i$ and hash  $j \in [\ell]$, we have
        \begin{align}
            \notag \E[Y_i]&=\E[ \sum_{k \in [U]-\{i\}}  \mathbf{I}(h_j(k)=h_j(i)) \cdot f_k(t)] \\
            & =   \sum_{k \in [U]-\{i\}} \frac{f_k(t)}{B} \leq \frac{t}{B}
        \end{align}
        
        According to the Markov's inequality
        \begin{align}
            \notag\mathbf{Pr}[Y_i \geq \epsilon \cdot t] &\leq \frac{\E[Y_i]}{(\epsilon \cdot t)^2} \\
            & \leq  \frac{t}{B\cdot (\epsilon \cdot t)^2} 
        \end{align}

        Plug $B=\frac{10}{\epsilon}$ to the above equation, we obtain $\mathbf{Pr}[Y_i \geq \epsilon \cdot t] \leq \frac{1}{10 \cdot \epsilon \cdot t}=0.1$.

        \item Prove that with probability $1 - \delta$, the algorithm guarantees that for every element $i$, at any moment $t$
        \begin{equation}
            \notag |CMS[j][h_j(i)]-f_j(t)| \le \epsilon \cdot t
        \end{equation}

        \Answer
        We first prove that for each element $i$, at any moment $t$
        \begin{equation}
            \mathbf{Pr}[|CMS[j][h_j(i)]-f_j(t)| \geq \epsilon \cdot t] \leq \frac{\delta}{N \cdot M}    
        \end{equation}
        
        Let $X_i \in \{0,1\}$ denote whether $|CMS[j][h_j(i)]-f_j(t)| \geq \epsilon \cdot t$ or not, for a given $j$. According to (a), we have
        \begin{equation}
            \E[X_i]\leq \frac{1}{10} \Rightarrow \E[\sum_{i=1}^n X_i] \leq \frac{\ell}{10}
        \end{equation}

        Let $\mu=\frac{\ell}{10}$, $\delta=9$. According to the Chernoff bounds, we have
        \begin{align}
            \mathbf{Pr}[\sum_{i=1}^n X_i \geq (1+9) \cdot \frac{\ell}{10}] \leq e^{-\frac{81\cdot l}{3 \cdot 10}} \notag \\
            \Rightarrow \mathbf{Pr}[\sum_{i=1}^n X_i \geq \ell] \leq e^{-\frac{81 \cdot \ell}{30}}
        \end{align}

        Since $\ell = \log_{2}(\frac{N\cdot M}{\delta})$, we have
        \begin{align}
            \mathbf{Pr}[\sum_{i=1}^n X_i \geq \ell] \leq e^{-\frac{81 \cdot \ell}{30}} \notag \\
            =(\frac{\delta}{N \cdot M})^{\frac{81}{30}} \leq \frac{\delta}{N \cdot M}
        \end{align}

        By applying the union bounds, we have

        \begin{align}
            &\mathbf{Pr}[|CMS[j][h_j(i)]-f_j(t)| \ge \epsilon \cdot t] \notag \\
            \leq &\sum_{i \in M, j \in N}\mathbf{Pr}[|CMS[j][h_j(i)]-f_j(t)| \ge \epsilon \cdot t] \notag \\
            =&N \cdot M \cdot \frac{\delta}{N \cdot M} = \delta 
        \end{align}

        Thus, $ \mathbf{Pr}[|CMS[j][h_j(i)]-f_j(t)| \le \epsilon \cdot t] = 1-\delta$.

        \item Compare the guarantee with the algorithm in lecture 3.
        
        \Answer
        The resulting guarantees are weaker however the space efficiency is higher due to incorrect frequency estimates.

        \item Discuss how to support \textsc{Deletion}$(i)$, which subtracts the frequency of $i$ by 1, and
        \textsc{ListHeavy}, which output a list of at most $O(1/\epsilon)$ elements that contain all heavy elements $i$ with $f_i(t) \geq \epsilon \cdot t$ like the set $A$ in the Misra-Gries algorithm.

        \Answer
        \textsc{Deletion}(i): Decrease $|CMS[j][h_j(i)]|$ by 1 for each $j \in [\ell]$.
        

        \begin{algorithm}[ht]
            \caption{ListHeavy}
            \begin{algorithmic}[1]
                \State Set $L= [ ]$.
            \For {each hash $j \in [\ell]$}
                \For {each $b \in [0, 1, ..., B-1]$}
                    \If {$CMS[j][b] \ge 2 \cdot \epsilon \cdot t$}
                        \State Insert $A[j][b]$ into $L$;
                    \EndIf
                \EndFor
            \EndFor \\
            \Return $L$
            \end{algorithmic}
            \end{algorithm}

            Since the element $A[j][b]$ contribute the most (\ie , $\geq \frac{1}{2}$) in $CMS[j][b]$, its true frequency $f_i(t) \geq \frac{1}{2}\cdot 2 \cdot \epsilon \cdot t = \epsilon \cdot t$. Besides, list $L$ has at most $\frac{1}{\epsilon}$ elements, since $\frac{t}{f_i(t)}\leq \frac{1}{\epsilon}$.
    \end{enumerate}
\end{problem}

\end{document}
